%% ----------------------------------------------------------------
%% Thesis.tex -- MAIN FILE (the one that you compile with LaTeX)
%% ---------------------------------------------------------------- 

% Set up the document
\documentclass[a4paper, 11pt, oneside]{Thesis}  % Use the "Thesis" style, based on the ECS Thesis style by Steve Gunn
\graphicspath{Figures/}  % Location of the graphics files (set up for graphics to be in PDF format)

% Include any extra LaTeX packages required
\usepackage[square, numbers, comma, sort&compress]{natbib}  % Use the "Natbib" style for the references in the Bibliography
\usepackage{verbatim}  % Needed for the "comment" environment to make LaTeX comments
\usepackage{vector}  % Allows "\bvec{}" and "\buvec{}" for "blackboard" style bold vectors in maths

\usepackage[utf8]{inputenc}

\usepackage{pdfpages}

\usepackage{graphicx}

% for tables
\usepackage{amsfonts}
\usepackage{booktabs}
\usepackage{siunitx}
\newcolumntype{L}[1]{>{\raggedright\arraybackslash}m{#1}}
\newcolumntype{C}[1]{>{\centering\arraybackslash}m{#1}}
\newcolumntype{R}[1]{>{\raggedleft\arraybackslash}m{#1}}

\hypersetup{urlcolor=blue, colorlinks=false}  % Colours hyperlinks in blue, but this can be distracting if there are many links.

% For code blocks
\usepackage{color}
\definecolor{mygreen}{rgb}{0,0.6,0}
\definecolor{mygray}{rgb}{0.5,0.5,0.5}
\definecolor{mymauve}{rgb}{0.58,0,0.82}

\lstset{ %
  backgroundcolor=\color{white},   % choose the background color; you must add \usepackage{color} or \usepackage{xcolor}
  basicstyle=\footnotesize,        % the size of the fonts that are used for the code
  breakatwhitespace=false,         % sets if automatic breaks should only happen at whitespace
  breaklines=true,                 % sets automatic line breaking
  captionpos=b,                    % sets the caption-position to bottom
  commentstyle=\color{mygreen},    % comment style
  deletekeywords={...},            % if you want to delete keywords from the given language
  escapeinside={\%*}{*)},          % if you want to add LaTeX within your code
  extendedchars=true,              % lets you use non-ASCII characters; for 8-bits encodings only, does not work with UTF-8
  frame=false,                    % adds a frame around the code
  keepspaces=true,                 % keeps spaces in text, useful for keeping indentation of code (possibly needs columns=flexible)
  keywordstyle=\color{blue},       % keyword style
  language=Octave,                 % the language of the code
  morekeywords={*,...},            % if you want to add more keywords to the set
  numbers=left,                    % where to put the line-numbers; possible values are (none, left, right)
  numbersep=5pt,                   % how far the line-numbers are from the code
  numberstyle=\tiny\color{mygray}, % the style that is used for the line-numbers
  rulecolor=\color{black},         % if not set, the frame-color may be changed on line-breaks within not-black text (e.g. comments (green here))
  showspaces=false,                % show spaces everywhere adding particular underscores; it overrides 'showstringspaces'
  showstringspaces=false,          % underline spaces within strings only
  showtabs=false,                  % show tabs within strings adding particular underscores
  stepnumber=1,                    % the step between two line-numbers. If it's 1, each line will be numbered
  stringstyle=\color{mymauve},     % string literal style
  tabsize=2,                       % sets default tabsize to 2 spaces
  title=\lstname                   % show the filename of files included with \lstinputlisting; also try caption instead of title
}

%% ----------------------------------------------------------------
\begin{document}
% ITU front page (required)
\includepdf[pages=1,
            pagecommand={\thispagestyle{empty}},
            %width=\textwidth,
            %height=\textheight,     
            keepaspectratio,
            %offset=0pt 0pt,     %letter, oneside    => misaligned by .x pt
            offset=60pt -60pt,  %letter, twoside    => misaligned by .x pt
            %offset=0pt 0pt,    %a4paper, oneside   => misaligned by 1.x pt
            %offset=-25pt 0pt,  %a4paper, twoside   => misaligned by 1.x pt
            ]{itufront.pdf}

\frontmatter      % Begin Roman style (i, ii, iii, iv...) page numbering

% Set up the Title Page
\title  {A music player user interface based on head gestures and 3D audio feedback}
\authors  {\texorpdfstring
            {\href{your web site or email address}{Jonas Hinge}}
            {Author Name}
            }
\addresses  {\groupname\\\deptname\\\univname}  % Do not change this here, instead these must be set in the "Thesis.cls" file, please look through it instead
\date       {\today}
\subject    {}
\keywords   {}

\maketitle
%% ----------------------------------------------------------------

\setstretch{1.3}  % It is better to have smaller font and larger line spacing than the other way round

% Define the page headers using the FancyHdr package and set up for one-sided printing
\fancyhead{}  % Clears all page headers and footers
\rhead{\thepage}  % Sets the right side header to show the page number
\lhead{}  % Clears the left side page header

\pagestyle{fancy}  % Finally, use the "fancy" page style to implement the FancyHdr headers


%% ----------------------------------------------------------------
% The "Funny Quote Page"
% \pagestyle{empty}  % No headers or footers for the following pages

% \null\vfill
% Now comes the "Funny Quote", written in italics
% \textit{``Write a funny quote here.''}

% \begin{flushright}
% If the quote is taken from someone, their name goes here
% \end{flushright}

% \vfill\vfill\vfill\vfill\vfill\vfill\null
% \clearpage  % Funny Quote page ended, start a new page
%% ----------------------------------------------------------------


% The Abstract Page
\addtotoc{Abstract}  % Add the "Abstract" page entry to the Contents
\abstract{
\addtocontents{toc}{\vspace{1em}}  % Add a gap in the Contents, for aesthetics

% Thomas comments
%regarding the abstract which you are about to write, i recommend you to have a look on the lecture slides from the lecture on "documenting your ubicomp project" lecture on the pervasive computing project course the past year. for a master thesis, the abstract should be half a page to a full page (shorter is better) and contain, in this specific order,

%1) a couple of sentences introducing the reader to the problem domain,
%2) ~one sentence highlighting the main problem you are adressing
%3) ~one sentence explaining your specific approach in adressing the problem
%4) ~one sentence describing your prototype system
%5) ~one sentence describing your final evaluation procedure and setup
%6) ~one sentence describing the main findings from the final evaluation
%7) ~one sentence that is a kind of outlook: what does the results from your work mean for the future?

}

\clearpage  % Abstract ended, start a new page
%% ----------------------------------------------------------------

\setstretch{1.3}  % Reset the line-spacing to 1.3 for body text (if it has changed)

% The Acknowledgements page, for thanking everyone
\acknowledgements{
\addtocontents{toc}{\vspace{1em}}  % Add a gap in the Contents, for aesthetics

The acknowledgements and the people to thank go here, don't forget to include your project advisor\ldots

}
\clearpage  % End of the Acknowledgements
%% ----------------------------------------------------------------

\pagestyle{fancy}  %The page style headers have been "empty" all this time, now use the "fancy" headers as defined before to bring them back


%% ----------------------------------------------------------------
\lhead{\emph{Contents}}  % Set the left side page header to "Contents"
\tableofcontents  % Write out the Table of Contents

%% ----------------------------------------------------------------
\lhead{\emph{List of Figures}}  % Set the left side page header to "List if Figures"
\listoffigures  % Write out the List of Figures

%% ----------------------------------------------------------------
\lhead{\emph{List of Tables}}  % Set the left side page header to "List of Tables"
\listoftables  % Write out the List of Tables

%% ----------------------------------------------------------------
%\setstretch{1.5}  % Set the line spacing to 1.5, this makes the following tables easier to read
%\clearpage  % Start a new page
%\lhead{\emph{Abbreviations}}  % Set the left side page header to "Abbreviations"
%\listofsymbols{ll}  % Include a list of Abbreviations (a table of two columns)
%{
% \textbf{Acronym} & \textbf{W}hat (it) \textbf{S}tands \textbf{F}or \\
%\textbf{HCI} & \textbf{H}uman \textbf{C}omputer \textbf{I}nteraction \\

%}

%% ----------------------------------------------------------------
% End of the pre-able, contents and lists of things

\setstretch{1.3}  % Return the line spacing back to 1.3

% \pagestyle{empty}  % Page style needs to be empty for this page
% \dedicatory{For/Dedicated to/To my\ldots}

\addtocontents{toc}{\vspace{2em}}  % Add a gap in the Contents, for aesthetics


%% ----------------------------------------------------------------
\mainmatter	  % Begin normal, numeric (1,2,3...) page numbering
\pagestyle{fancy}  % Return the page headers back to the "fancy" style

% Include the chapters of the thesis, as separate files
% Just uncomment the lines as you write the chapters

\lhead{\emph{Introduction}}
\chapter{Introduction}
% Mobile interface design problem
Mobile and wearable devices has been a growing area in computing in recent years. Compaired to desktop computers these devices have introduced new standards for when and how people interact with especially mobile applications. Suddenly people are able to check the news, navigate via interactive maps, post social messages, listen to music, etc., while they are on the move. However, most mobile systems are not designed for interaction in motion \cite{marshall_mobile_2013}. Instead they are primarily based around users stopping and interacting with the device. This is caused by the main communication channel between the user and device - the touchscreen interface. This interface requires the eyes of the users to perceive the information displayed on the screen and at least one hand for touch-based input. Many applications delivers information through non-visual communications channels like audio. In many cases however this information is presented in a one way transfer approach e.g. navigation instructions, music or the audio of an incoming phonecall where users still needs to interact with the touchscreen to respond e.g. type in destination location, answer the phonecall or switch music track. To solve this many especially older devices have dedicated buttons for such respond tasks and also many headsets include controllers (attached on the wire of the headset). However though such physical buttons can be used while moving without looking at any screen, the number of actions to perform are often quite limited like in the PocketMenu \cite{pielot_pocketmenu:_2012} which is a music player interface that provides "pocket interaction" with simple controls like volume, next/previous track, play and pause.

% Usage of this interface problems
It can not be assumed that people stop and visually attend a touchscreen interface everytime they wish to interact. In most cases people use the "stop-to-interact" design while they are in motion making interaction awkward and challenging. People can not easily devote their visual attention to an interface while walking, running or driving \cite{brewster_overcoming_2002}. For instance, although screen resolutions and physical sizes of mobile devices are increasing, the visual work space is limited i.e. screens easily becomes cluttered with information and the input keyboard can be challenging when moving e.g. small buttons or non-responding touch interfaces. More importantly, when moving around e.g. in the traffic, interacting with a mobile device in this way could reduce the users ability to perceive and react to the surroundings because of the eyes or hands being occupied by the touchscreen. In the worst case this would cause accidents. Motivated by this problem fines are introduced for example in Denmark for people interacting with their mobile device while biking\footnote{\url{http://www.cyklistforbundet.dk/Alt-om-cykling/Love-og-regler/boedetakster}}. This could indicate that hands-free mobile interfaces will solve the problem. Much of the interface work in wearable computing are focusing on visual displays, often presented through headmounted displays \cite{barfield_fundamentals_2000}. However these interfaces can be hard to use in bright daylight and they occupy the users visual attention \cite{geelhoed_safety_2000}. Also that an interface is eyes and hands-free doesn't neccesarily mean that it improves other human aspects like cognitive load. For example interaction through auditory interfaces e.g. hands-free phone talking while driving, is prohibited in several countries.

% Research
Several research areas in the HCI community have targeted to support interaction while in motion. Using alternative and multiple interaction modalities like speech, gesture and gaze tracking (multimodal interaction) can not only help developing hands- or eyes-free interfaces but it has also been shown to decrease the perception and cognition problems mentioned above.

\newpage

\section{Research Questions}
\label{sec:researchquestions}
Using mobile devices while moving around in physical demanding environments implies extra cognitive and perceptual load and there exists only few systems today that are designed to handle this. However interfaces that allows communication with mobile applications through other modalities than touch and vision are showing up like the Intelligent Headset\footnote{\url{https://intelligentheadset.com/}} which is based on head gestures and 3D audio.

% Question
Inspired by the fact that such interface could liberate the eyes and hands when interacting while in motion and that gestures and audio modalities have shown to improve usability of mobile devices under eyes-free mobile conditions \cite{brewster_multimodaleyes-freeinteraction_2003} - the following question is asked: Can a user interface based on head gestures and 3D audio compete with existing "stop-to-interact" user interfaces (touch and vision-based) with respect to:

\begin{description}
\item[1] General usability (cognitive/perceptual load).
\item[2] Task performance efficiency (navigation and exploration of application content).
\end{description}


\section{Goal}
\label{sec:goal}
To answer the research questions we will look at a very common scenario where people interact with a device, while they are performing a physical activity that would require the use of eyes and hands: Biking while listening to music - or more concretely the subscenario: Biking while switching music track.

This scenario is inspired from the fact that the leading mobile music applications today are designed by the "stop-to-interact" model and although there exists eyes-free music player interfaces these are limited to simple controls. The goals of this thesis are defined as follows:

\begin{description}
\item[1] Design and develop a mobile music player interface based on head gestures and 3D audio that allows a user to explore and select music tracks.
\item[2] Evaluate the final user interface in a simulated biking scenario and compare it with an existing touch and vision-based music player user interface in terms of general usability and task performance efficiency.
\end{description}


\section{Method}
In short the method is to design, implement and evaluate a system that achieves the goals of this project. We start in chapter \ref{sec:relatedwork} by researching 3 areas; Mobile HCI, Multimodal Interaction and Mobile Audio Interfaces, which all are related to this thesis focus. The related systems designs from \ref{sec:mobileaudiointerfaces} will, by comparing with thesis goals and related research, inspire to a first system design. By taking a human-centred approach \cite{benyon_designing_2010}, users will evaluate the first system design and this will enable specifications to emerge and lead to a final design. To help users to understand the system a hi-fi prototype is used for envisionment \cite{benyon_designing_2010} during the design process. The system design and process is described in chapter \ref{sec:design}. The final system is evaluated in a controlled lab experiment and is described in chapter \ref{sec:evaluation}. The thesis process and activities are shown in figure \ref{fig:triangulation}.

\begin{figure}[t]
	\centering
		\includegraphics[width=\textwidth,height=\textheight,keepaspectratio]{./Figures/triangulation.pdf}
		\rule{35em}{0.5pt}
	\caption[Triangulation]{Mapping of thesis activities and process using the triangulation framework proposed by Mackay and Fayard \cite{mackay_hci_1997}}
	\label{fig:triangulation}
\end{figure}








% OLD STUFF

%As the usage of mobile applications are increasing\footnote{\url{http://mashable.com/2014/01/14/mobile-app-use-2013/}} - 113\% in 2013 - the need for better system designs grows. 

%Fines are also being introduced in the U.S. for texting while biking e.g. in California\footnote{\url{http://blogs.lawyers.com/2011/08/there-oughta-be-a-law-california-may-ban-texting-while-biking/?test=400}} and Charleston\footnote{\url{http://www.postandcourier.com/apps/pbcs.dll/article?AID=/20131008/PC16/131009426}}. As Charleston law suggests there are exceptions: \textit{"The exception would be for a device that can be worked hands-free."}.

%This scenario is inspired from the fact that mobile music listening on smartphones has increased dramatically\footnote{\url{http://www.emarketer.com/Article/Music-Goes-Mobile-More-Smartphone-Users-Stream-Songs/1010126}} and the leading music streaming smartphone applications today are designed by the "stop-to-interact" model. The goals of this thesis are defined as follows:

%Motivation:
%... audio feedback can decrease cognitive load
%... spatial audio can deliver more advanced feedback
%... current hands-free music players are limited in control features

%A very common scenario where people interact with a mobile device while in motion is when they are biking and listening to music. At the same time music listening on smartphones has increased dramatically with the emerging music streaming services\footnote{\url{http://www.emarketer.com/Article/Music-Goes-Mobile-More-Smartphone-Users-Stream-Songs/1010126}} and the leading mobile music streaming applications today are designed by the stop-to-interact model (touch and vision-based).

%Interfaces that uses other modalities than touch and vision are showing up like the Intelligent Headset\footnote{\url{}} which can detect a users head movements and give 3D audio feedback. Such an interface could potentially liberate the eyes and hands else required by the touchscreen.

%\subsection{Research questions}

%The goal of this thesis is to test if an interface like the Intelligent Headset is feasible in a mobile music application and 


%The goal of this thesis is to design and evaluate a mobile music player that uses the Intelligent Headset that can compete with existing user interfaces for music players (e.g. touch and vision-based).

%Considering interacting with a mobile device while in motion, this project will be based on the concrete scenario where people are biking while listening to and controlling their music libray. As biking requires eyes on the road and hands for steering the input/output modalities should preferrably not include eyes and hands. Instead head gestures for input and 3D audio for output will be evaluated.

%More specifically the following questions should be answered. Can a user interface based on head gestures and 3D audio compete with existing user interfaces for music players (e.g. touch and vision-based) with respect to for instance:
%\begin{description}
%\item[1] Navigating content (exploring music tracks)
%\item[2] General usability (cognitive/perceptive load)
%\item[3] Suitability to real-world hands-occupied situations
%\end{description}
%Furthermore could an interface using the chosen input and output modalities increase a users awareness of the surroundings i.e. improve the safety, when biking in a trafficked environment?

%TODO: Something about exploring the music content? Compaired to not just switching track (next/prev button)

% OLD
%More specifically the following questions should be answered. Can a user interface based on head gestures and 3D audio compete with existing user interfaces for music players (e.g. touch and vision-based) with respect to for instance:
%\begin{description}
%\item[1] Navigation and control efficiency
%\item[2] Learnability
%\item[3] General usability (cognitive/perceptive load)
%\item[4] Suitability to real-world hands-occupied situations
%\end{description}
%With the chosen combination of input and output modalities, there is a high risk for the system to misinterpret normal everyday actions performed by the user as commands for controlling the system ("behavioural cluttering" (Janlert et al., in press)). How can features in the user interface prevent unwanted manipulation of the system?

%\section{Goal}
%To measure properties from the problem statement the goal of this thesis will be to:
%\begin{itemize}
%\item Design an interface that can detect head movements and provide audio feedback and at the same time is appropriate in the concrete mobile scenario where people are biking.
%\item Design and implement music player software that can handle data from the interface and present this in form of a user friendly navigable menu.
%\item Imperically compare the new music player with an existing one and gather information on whether the developed system possibly can increase safety when biking in traffic i.e. make people more aware of what is going on around them while they are biking and navigating the system.
%\end{itemize}

% OLD
%The goal of this project is to examine if head gesture based input and audio output modalities in combination can compete with a traditional touch and vision based input/output interface and show which advantages, disadvantages and challenges that arise when designing and using such interaction techniques. More precisely a mobile system that recognises these alternative interaction methods should be designed and implemented in a music application. To measure properties from the problem statement, the final system should contain:
%\begin{description}
%\item[1] Music menu navigation
%\item[2] Head gestures recognition
%\item[3] Menu items in a users 3D audio space
%\end{description}
%Such a system should be evaluated in a scenario where the user interacts while in motion e.g. a biking scenario.


% OLD
%In the final evaluation users will compaire this new way of controlling a music application with a traditional music application in form of usbility, efficiency, learnability and suitability. This will happen in a closed lab test where users should bike while navigating the system.


%At the same time emerging hardware e.g. sensor technology in mobile devices and wearable computing expands the user interaction possibilities.

%Challenges arise when interacting with mobile devices. Although screen resolutions and physical sizes of mobile devices are increasing, the visual work space is limited i.e. screens easily becomes cluttered with information and the input keyboard can be challenging when moving e.g. small buttons or non-responding touch interfaces. More importantly, when moving around e.g. in the traffic, interacting with a mobile device at the same time can create challenges in form of distractions e.g. "eyes off the road" or "hands occupied" and in the worst case cause accidents. Motivated by this problem fines are introduced (in Denmark) for people interacting with their mobile device while biking\footnote{\url{http://www.cyklistforbundet.dk/Alt-om-cykling/Love-og-regler/boedetakster}}. Fines are also being introduced in the U.S. for texting while biking e.g. in California\footnote{\url{http://blogs.lawyers.com/2011/08/there-oughta-be-a-law-california-may-ban-texting-while-biking/?test=400}} and Charleston\footnote{\url{http://www.postandcourier.com/apps/pbcs.dll/article?AID=/20131008/PC16/131009426}}. As Charleston law suggests there are exceptions: \textit{"The exception would be for a device that can be worked hands-free."}.

%So it seems that solutions to this problem could be found in the area of "interacting while in motion". The emerging hardware (e.g. sensor technology) and software opens up for alternative input modalities e.g. head gestures, gaze tracking, speech recognition making hands-free interaction possible. At the same time output modalities such as audio and haptic feedback could liberate the eyes from the screen.


% Thomas comments
% - General problem, mobile devices can not be configured while on the move, then move on to the specific biking problem, stick to just mentioning the problem, interaction on the move statement ref to paper, the goal of this thesis is to design a user interface that uses other modalities while on the move and interacting
% - be general, do not mention music scenario
% - In goal description be more specific, bike scenario, maybe mention headset in the goal
% - make a bridge between problem statement and goal, start with: "a very common scenario is biking and listening to music..." maybe find some statistical info
% - goal: list exactly what to measure, put something about safety in the goal description


%\section{Problem statement}
%Using mobile devices while moving around in physical demanding environments implies extra cognitive and perceptual load and there exists only few systems today that are designed to handle this. 

% However people still use these systems while they are on the move, despite the awkward interaction. This will reduce peoples attention i.e. reduce safety when moving around in traffic, and as the usage of mobile applications are increasing\footnote{\url{http://mashable.com/2014/01/14/mobile-app-use-2013/}} - 113\% in 2013 - the need for better system designs grows. Some systems out there support eyes-free interaction but they are only providing simple static controls like the PocketMenu system \cite{pielot_pocketmenu:_2012} which limits the use of application features.




 % Introduction

\lhead{\emph{Research}}
\chapter{Research}
TODO: intro


\section{Research areas}
This section will give an overview of this projects relevant research areas within the HCI paradigm. Especially the focus should be within this projects main goals namely in the mobile HCI area with two interaction styles in mind: Hands-free interaction and eyes-free interaction. A graphical overview of this is presented in fig. \ref{fig:venn} and the focus areas are described.

\begin{figure}[htbp]
	\centering
		\includegraphics[width=\textwidth,height=\textheight,keepaspectratio]{./Figures/venn-diagram.pdf}
		\rule{35em}{0.5pt}
	\caption[Venn diagram]{A comparison of thesis topics}
	\label{fig:venn}
\end{figure}

\subsection{Human Computer Interaction}
...

\subsubsection{Mobile HCI}
...

\subsection{Hands-free interaction}
This term refers to controlling a system without using the hands including no hand gestures although sometimes used this way e.g. refers to not holding a device in the hand \cite{witt_designing_2006}. Achieving this hands-freeness interaction often relates to speech recognition or gaze/head tracking techniques but also other body parts are used for simple interactions e.g. leg shifting music track while running \cite{smus_running_2010}. Speech recognition is becoming a more common interaction modality but there exists accuracy and stability challenges especially in mobile noisy environments. In contrast head tracking could possibly provide a more stable interaction detection in mobile environments.

\subsection{Eyes-free interaction}
Several work on both audio \cite{kajastila_eyes-free_2013,bonner_no-look_2010,brewster_multimodaleyes-freeinteraction_2003,zhao_earpod:_2007,vazquez-alvarez_eyes-free_2011} and haptic \cite{pasquero_haptic_2011,pielot_tactile_2011} displays use the term eyes-free which refers to controlling the state of a system without visual attention. This kind of interaction has shown to be desirable in some situations \cite{oakley_designing_2007,yi_exploring_2012} and even improve efficiency compaired to traditional visual displays \cite{zhao_earpod:_2007}.

% visual competition, concentration
One of the main motivations behind this eyes-free use is to design interfaces that do not compete with the users visual attention. That is this "visual competition" could introduce risks when people are on the move e.g. travelling in traffic. In these situations a vital factor is to minimize the amount of distraction for interaction modes \cite{pascoe_using_2000}. Eyes-free interfaces can keep the users visual attention on the road while driving \cite{sodnik_user_2008} or walking around in the city \cite{vazquez-alvarez_eyes-free_2011}.

% visual display problems
Much of the interfaces work in wearable computing tends to focus on visual headmounted displays \cite{barfield_fundamentals_2000} e.g. Google Project Glass. But not only as mentioned does visual displays occupy the users visual attention, they can also be obtrusive and hard to use in bright daylight \cite{geelhoed_safety_2000}. Another disadvantage with visual displays is that their power consumption is high i.e. they drain a mobile device battery and they are expensive. By using eyes-free interfaces it is possible to use cheaper and less power consuming hardware.


\section{Spatial sound}
In progress...

(Spatial audio, Head Related Transfer Function)...\\
Good reference for 3d sound \cite{begault_3dd_1994}


\section{Head gestures}
There exists different kinds of areas when it comes to controlling a system with head gestures. Using cameras it is possible to effectively track head movements via facial recognition \cite{morimoto_recognition_1996} and gaze tracking makes it possible to control an object by fixating the eyes on that object while moving the head \cite{mardanbegi_eye-based_2012}. Thus these techniques do not require any hardware sensors e.g. accelerometer and gyroscope but in return a camera placed in front of the user. This will constrain the use especially in mobile "on-the-move" situations.

\subsection{Motion gesture recognition}

Dynamic Time Warping \cite{salvador_toward_2007}

Accelerometer-based DTW \cite{akl_accelerometer-based_2010}

\subsection{Intelligent Headset}
...


\section{Related work}
TODO: intro

% closely related to my project
Brewster et al. showed that novel interaction techniques based on sound and gesture can significantly improve the usability of a wearable device in particular under "eyes-free" mobile conditions and that head gestures was a successful interaction technique with egocentric sounds the most effective \cite{brewster_multimodaleyes-freeinteraction_2003}.

Park et al. also experimented, using head gesture input and aural output, with 1D and 2D menu interfaces \cite{park_gaze-directed_2011}.

Kajastila and Lokki has done a user study comparing auditory and visual menus controlled by the same free-hand gestures where the majority of the participants felt that an auditory circular menu was faster than a visual based menu \cite{kajastila_interaction_2013}.

% maybe more spatial sound oriented
William W. Gaver, a pioneer in audio interfaces, has explored several aspects of using sound in interfaces including the intuitiveness of presenting complex information to users in the form of audio \cite{gaver_sonicfinder:_1989}. Similarly Graham explores the advantages in reaction time when using ”auditory icons” \cite{graham_use_1999}. In \cite{gaver_auditory_1986} Gaver presents the use of spatial sound icons. In doing so, he draws forward the unutilized potential of creating natural interaction through spatial audio.

Work has shown that non-speech audio is effective in improving the interaction with mobile devices \cite{pirhonen_gestural_2002, sawhney_nomadic_2000}.

By compairing visual and audio feedback when pushing buttons on the same GUI, Brewster showed that it was difficult for users to devote all their visual attention to an interface while walking, running og driving and that the interaction workload decreased with audio feedback \cite{brewster_overcoming_2002}.

\subsection{Compairing related work}
Summing up background (project focus).

Table compairing properties of related work (and this project) example fig. \ref{tab:related} (NOTE: temp, i need some input for this...).

\begin{table}[h] 
\caption{Related works properties comparison} % title name of the table 
%\centering % centering table

\begin{tabular}{L{4cm}C{2cm}C{2cm}C{2cm}C{2cm}} \toprule
    Related work & Head gesture & Spatial sound & Music application & Accessible hardware \\ \midrule
    Multimodal eyes-free interaction techniques for wearable devices \cite{brewster_multimodaleyes-freeinteraction_2003}  & + & + & - & - \\ \midrule
    This project  & + & + & + & + \\ \bottomrule
\end{tabular}

\label{tab:related} 
\end{table}










 % Background

\lhead{\emph{Design}}
\chapter{Design}
In this chapter the design of the system is described. This is done by first describing a design rationale based on the concepts and related work from chapter \ref{sec:relatedwork} and next how the actual system was designed.

\section{Design Rationale}

% From wiki:
% - the reasons behind a design decision,
% - the justification for it,
% - the other alternatives considered,
% - the trade offs evaluated, and
% - the argumentation that led to the decision.

% Intro
Before choosing a design for a system different factors needs to be considered e.g. the reasons behind design decisions and the justification for it; other design alternatives considerations and the tradeoffs when choosing a design over another; the argumentation that lead to the design.

% Ubicomp challenges
When designing for ubicomp new aspects needs to be taken into consideration which increase the complexity of the system e.g. different devices, mobile users and changing environment and context \cite{barfield_fundamentals_2000}. Also we note that we are not only designing user interfaces but also the interactions between a user and the system through artifacts embedded in the environment \cite{beaudouin-lafon_designing_2004}. Furthermore it should be taken into account whether the interaction happens whilst the user is in motion as this introduces even more complexity as decribed earlier in section \ref{sec:interactioninmotion}.

In this project we have defined the following requirements of the system: A user should navigate a music player using head gestures and audio feedback while biking. In other words we have a user interacting with artifacts, consisting of a headmounted interface and a mobile device, while biking in a traffic environment implying user awareness of road conditions, cars, other bikers, etc.

\subsection{Devices}
- Head gestures, sensors on head, headset for audio, mention gaze tracking solution

\subsection{User}
- Physical constrains: Head rotation, Human head normally can be rotated about 140 degrees for shaking and 100 degrees for nodding \cite{lopresti_neck_2000}
- Workload
- Perceptive load
- Mental model

\subsection{Environment}
- Unintentional behaviour e.g. activating menu biking over a bump or a sudden need for rotating head because of obstacles
- Weather conditions, too much wind disrupting the audio, rain on headset
- Noise can disrupt audio


\section{Spatial Music Menu}

\subsection{Soundscape}
- Music, strength of recognizing artist/track through listening vs seeing the text on a screen
- Horizontal argument
- Simultanous sounds, exploring, cocktail party effect argument
- Experimental design, sounds perceived, zoom effect, user should detect sound direction (which track)

Metaphor: Carousel

Soundscape design illustration, figure \ref{fig:sounddesign}

\begin{figure}[htbp]
	\centering
		\includegraphics[width=0.6\textwidth,height=\textheight,keepaspectratio]{./Figures/sounddesign.png}
		\rule{35em}{0.5pt}
	\caption[Soundscape Design]{Soundscape Design - Visualising how the circular auditory menu surrounds a person (viewed from the persons back)}
	\label{fig:sounddesign}
\end{figure}

\subsection{Navigation}
- Adaption of real world music player menu
- Figure showing levels of menu
- Navigation
- Head gestures
- Activating menu
- Feedback when gesture recognized
- Nod = yes

Several studies show that circular auditory menus are the way to go because of horizontally positioned sounds (ref?)


\section{Experimental Prototype}
...

Challenge: introducing untraditional way of interacting, human-centred approach

Two menu interaction modes, distance/


























% OLD - Use some parts here and move some to introduction method section
\begin{comment}

This chapter first explains the design methods used and the most important design activities and choices made in the design process. Based on this process the final prototype design is presented at the end of the chapter.

\section{Design model and methods}
In this section the model used for designing the final prototype is presented including the different techniques used throughout the design process.

The related works conducts the foundations for an early first prototype. This prototype will then go through an iterative design process, taking a user centered approach. This will enable specifications to emerge during the process and these learnings and modifications will result in new experiments and prototypes. This iterative design model is illustrated in figure \ref{fig:iterative}.

\begin{figure}[htbp]
	\centering
		\includegraphics[width=0.6\textwidth,height=\textheight,keepaspectratio]{./Figures/iterative.png}
		\rule{35em}{0.5pt}
	\caption[Iterative Design Model]{Iterative Design Model}
	\label{fig:iterative}
\end{figure}

\subsection{Envisionment}
One of the goals with the final system is that it should be eyes-free i.e. the user should not depend on a visual screen UI at the end. Despite this goal, the use of a screen to visualize e.g. a virtual menu, can be helpful in the design process. This will enable test users to get a quicker and better understanding of how the interaction works. In this design process a mobile device screen will be used for envisionment - this is also called a hifi prototype or software prototype \cite{benyon_designing_2010}.

More concretely audio sources and the users head position including rotation will be mapped visually to a screen dynamically during user interaction. An example of this screen is shown in figure ?

\section{Design process}
This section starts out by describing the first experimental prototype design inspired from previous research work and then each iteration including user feedback and design changes and experiments. More specifically the design completed 2 iterations before the final prototype design.

\subsection{Initial prototype design considerations}
Before we started to reason about initial design choices we started out by defining what the system actually needed in order to evaluate and revise the hypothesis from the problem statement.

% intro, track exploring
First of all we wanted the system to be able to play music. This is in itself a trivial task but as we the same time wanted to control the system only with with head gestures and audio output a traditional music player includes too many options e.g. play, pause, stop, next/previous, volume, equalizer, track exploring, etc for the scope of this thesis. All the alternative music players mentioned in chapter \ref{sec:relatedwork} is limited to simple commands like play, stop, next/previous and volume. Taking it a step further we wanted to evaluate the track exploring part i.e. navigating to a preferred track and playing it. This made et clear that some kind of auditory menu with tracks as menu items was needed and although not music players the related systems from chapter \ref{sec:relatedwork} using auditory menus could inspire to an intital menu design.

% Auditory menu, exocentric, head rotation constraints
When looking at the different related auditory menu designs we needed to find a design that fitted into the context of the user activity i.e. biking and also user interaction modality. E.g. in a biking scenario the user should have eyes on the road thereby constraining the head rotation. Park et. al \cite{park_gaze-directed_2011} showed good results with a 2D grid menu. It should be taken into account that their audio output consists of simple speech commands e.g. speech recorded numbers. We want to present multiple music streams (non-speech audio) and studies have shown that when presenting multiple non-speech audio streams simultanously in a spatial audio space, segregating the audio streams horizontically has a better effect than vertical alignment [TODO: ref]. It seemed that a circular auditory menu could be a good starting point and both Kajastila and Lokki \cite{kajastila_interaction_2013} and also Brewster et. al \cite{brewster_multimodaleyes-freeinteraction_2003} evaluated this kind of menu design with good results.

[TODO: Image/prototype sketch?]

% Circular menu design, head gestures
While placing music streams in a circular way around a users audio space we needed a way of navigating to and chosing a specific track. The Brewster et. al \cite{brewster_multimodaleyes-freeinteraction_2003} system uses directed nods to choose an item but this limits the number of items to 4 (1 for every 90 degrees) as a nod in 45 degrees

The inititial interaction design was inspired from Kajastila and Lokkis system although they used hand gestures as modality input. 

For navigating and choosing menu items

[TODO: egocentric vs exocentric audio output, Brewster system good ref \cite{vazquez-alvarez_eyes-free_2011}]

% Summarising

[TODO]
In this project multiple audio sources (music tracks) are presented for the user at the same time but none of them requires a respond i.e. the focus is on selective-attention tasks.

\end{comment}







 % Design

\chapter{Evaluation}
...

% Iterations, measurable comparison between new system and traditional

% 2 evaluations - closed lab (1 day) and open (real life, week(s))

% Idea for closed lab exercise - Multiple lists of songs. A user shouls navigate and play the different songs with head gestures and normal navigation. Compare these in relation to time taken, cognitive load (eyes and at least one hand occupied), user feel of frustration (cognitive load) when navigating

% Final evaluation:
% Idea: Time to find a song, level of frustration (cognitive load) for finding song

% NB: For final evaluation - device with 3G+ connection and added to apple developer team, should be executed latest mid of April so finished end of April (1/2 weeks trial), 2 testpersons -> 1 experienced tech person and 1 non-tech/average user % Implementation

\lhead{\emph{Evaluation}}
\chapter{Evaluation}
% Intro
As mentioned in section [TODO: ref implementation section] the scope of the implementation in this thesis is limited to study the effects of exploring and navigating to music tracks. To study whether the Spatial Music Menu could compete with a touch and vision-based music player interface we designed and conducted an experiment where users should perform mental and physical demanding tasks while interacting with the interfaces.

[TODO: rewrite]

- Only 5 participants, no statistical arguments but just hints
Regarding

- Possibly biased participants (friends) wants to perform good

- Limited measurement parameters e.g. not biking speed


\section{Experiment Design}
The experiment was designed and conducted as a controlled lab experiment. Controlled experiments are appropriate when comparing one design to another to see which is better \cite{benyon_designing_2010} and in this case we are compairing the Spatial Music Menu with a touch and vision-based music player. As the focus is on compairing and study the effects of these interfaces in an interaction in motion scenario i.e. biking, we designed a evaluation system simulating a trafficked biking scenario.

\subsection{Biking simulation setup}
A stationary bike was put in front of a giant screen (4 x 40 inch HD screens) that should simulate a road. To make the view as realistic as possible an image of an actual trafficked road\footnote{New York street: \url{http://timsklyarov.com/new-york-through-the-eyes-of-a-road-bicycle/}} was showed on the screen. To simulate obstacles that the user should be aware of or respond to in a real world biking scenario, 3 different shapes with random colors were displayed in random positions on the screen in a random time interval; between 0.3 and 1 second displaying the shape in 0.8 seconds. The shapes were circles, triangles and squares and the job for the person riding the bike was to detect the circles. This was done by pushing a button attached close to the users non-preferred hand on the steer, in this case a Playstation 3 joystick strapped with tape. The reason for the button placement at the non-preferred hand was, that the preferred hand should be used for navigating the touch and vision-based music player. To give the user feedback the circle was removed when detected. The screen simulation software is developed in Python 3 running on a Mac (OSX 10.9). The simulation system is illustrated in figure [TODO: ref].

[TODO: figure of biking simulation system]

\subsection{Touch and vision-based music player}
To represent a touch and vision-based music player we chose the music streaming service Deezer\footnote{\url{https://www.deezer.com/}}. Their Android application\footnote{\url{https://play.google.com/store/apps/details?id=deezer.android.app}} was installed on a Google Nexus 4 running Android 4.3. The same headset as for the Spatial Music Menu was used but this time connected through a wire.

\subsection{Hypotheses}
Based on the related work theory behind the design choices for the Spatial Music Menu we derived the following hypotheses for the experiment:

\begin{description}
\item[Hypothesis 1:] The users ability to detect circles while executing tasks in the biking simulation will increase with the Spatial Music Menu interface compaired to the touch and vision-based music player interface.
\end{description}

\begin{description}
\item[Hypothesis 2:] The users performance when executing tasks in the biking simulation using the Spatial Music Menu can compete with the touch and vision-based music player interface in terms of general usability (workload).
\end{description}


\section{Method}
[TODO: intro]

\section{Participants}
5 persons (all male) were chosen for the evaluation. They all have in common that they listens to music while biking regurarly. The participants had an average age of 30 years.

\subsection{Procedure}
[TODO: instructions, preparation, favourite tracks]
% Prerequisites
... before starting the experiment the user were instructed in how the Spatial Music Menu works i.e. which head gestures to use for navigating and how the menu structure looks. They then got a chance to try out the system both standing still and while riding the stationary bike. In the standing still scenario the user was allowed to look at the menu envisioned on the iPad screen to get a sense of the menu structure and interaction feedback...

% Task description
[TODO: Task description]

- Artist - Album - Track

- task traditional: Pick phone from pocket, activate, navigate, deactivate, back in pocket

- choosing music tracks in Deezer app

User performing a task using the Spatial Music Menu is shown in figure \ref{fig:evalspatial}

\begin{figure}[htbp]
	\centering
		\includegraphics[width=0.7\textwidth,height=\textheight,keepaspectratio]{./Figures/evaluation_spatial.jpg}
		\rule{35em}{1pt}
	\caption[Evaluation Spatial Music Menu]{Participant performing a task using the Spatial Music Menu}
	\label{fig:evalspatial}
\end{figure}

User performing a task using a touch and vision-based music player is shown in figure \ref{fig:evalspatial}

\begin{figure}[htbp]
	\centering
		\includegraphics[width=0.7\textwidth,height=\textheight,keepaspectratio]{./Figures/evaluation_normal.jpg}
		\rule{35em}{1pt}
	\caption[Evaluation touch and vision-based interface]{Participant performing a task using a touch and vision-based music player}
	\label{fig:evalspatial}
\end{figure}

\subsection{Logging}
Throughout the experiment several data were logged. This includes data from the Biking Simulation System: Task start/end, circles shown, circles detections, error detections - and data from the Spatial Music Menu: Gestures detected, navigation steps including track info, headset connection status. Every log subject has a timestamp and as logging were performed on two different systems - iOS and OSX (python script) clocks were synchronized before comparison.


\subsection{NASA Task Load Index}
Subjective workload was measured using the NASA Task Load Index (TLX) scales \cite{hart_workload_1990}. The scales includes mental demand, physical demand, temporal demand, performance, effort and frustration in which each participant should rate after testing. The user ratings gives a qualitative analysis of the system and the perceived workload is linked with the general usability of the system.

\subsection{Observing and semi-structured interview}

- Monitoring interaction via iPad screen, calibrating center direction

- General user feedback at the end of tests


\section{Results}
[TODO: intro]

\subsection{User performance}
[TODO]
% Number of circles, safety
%The most important task for the user during testing is to detect as many circles as possible. The number of circles shown and the number of circles detected were logged during execution of tasks. This will give a quantitative analysis of whether the user is able to monitor the surroundings while interacting with the systems and contribute to the safety problem focus.

% System, track exploring
%For measuring the content (music track) exploring part of the Spatial Music Menu all navigation information were logged on the iPad. 

% System, task time

Circles detected, figure \ref{fig:resultscircles}

\begin{figure}[htbp]
	\centering
		\includegraphics[width=0.9\textwidth,height=\textheight,keepaspectratio]{./Figures/results_circles.png}
		\rule{35em}{1pt}
	\caption[Results circle detections]{Percentage of circles detected for the participants}
	\label{fig:resultscircles}
\end{figure}


Task time execution, figure \ref{fig:resultstasktime}

\begin{figure}[htbp]
	\centering
		\includegraphics[width=0.9\textwidth,height=\textheight,keepaspectratio]{./Figures/results_task_time.png}
		\rule{35em}{1pt}
	\caption[Results task time]{Time taken (in seconds) in average to execute a task for the participants}
	\label{fig:resultstasktime}
\end{figure}

\subsection{Workload}
[TODO]

... it can give an indication of how much of the users physical and mental resources is required by the system during interaction and thereby an indication of the resources left for monitoring and navigating the surroundings when biking i.e. a safety parameter.

Nasa TLX scores, figure \ref{fig:resultsnasatlx}

\begin{figure}[htbp]
	\centering
		\includegraphics[width=0.9\textwidth,height=\textheight,keepaspectratio]{./Figures/results_nasatlx.png}
		\rule{35em}{1pt}
	\caption[Results NASA TLX Score]{NASA TLX overall workload score for the participants (less is better)}
	\label{fig:resultsnasatlx}
\end{figure}

\subsection{Discussion}
...





% Maybe comfort as a measurement?
%(Taken from Brewster article)
%The final measure taken was comfort. This was based around a new scale developed by Knight et al. [10] called the Comfort Rating Scale (CRS) which assesses various aspects to do with the comfort of a wearable device. For a device to be accepted and used it needs to be comfortable and people need to be happy to wear it. Using a range of 20- point rating scales similar to NASA TLX, CRS breaks com- fort into 6 categories: emotion, attachment, harm, perceived change, movement and anxiety. Knight et al. have used it to assess the comfort of two wearable devices they are building in their research group. Using this will allow us to find out more about the actual acceptability our systems.



 % Evaluation

\lhead{\emph{Conclusion}}
\chapter{Conclusion}
% intro
In this thesis we have examined whether a mobile music player interface based on head gestures and 3D audio, could compete with a touch and vision-based music player in an "interaction in motion" scenario i.e. biking, with respect to general usability and task performance efficiency.

% related work
We started by investigating two main research areas: Mobile HCI and multimodal interaction. The mobile HCI research helped us understand the challenges that exists when people are interacting while in motion and that most mobile systems today are designed by the "stop-to-interact" model. The multimodal interaction litterature clarified the definition of modalities and their use especially in mobile systems. We delved deeper into the predefined modalities of our system; Head gestures and audio including spatial audio. We also investigated the current state on mobile audio user interfaces which included todays mobil music player interfaces and other related systems that uses auditory gesture-based interfaces.

% design, implementation
We continued by designing and implementing the Spatial Music Menu, an auditory menu interface designed for exploring and selecting music tracks using head gestures and spatial audio feedback. The system consisted of a headset (the Intelligent Headset, section \ref{sec:implementationheadset}) and an iPad running the iOS application. The design of the Spatial Music Menu was initially inspired from related systems and user feedback during the design process enabled specific features and adjustments to the system. 

% evaluation, discussion
A biking simulation system was designed to evaluate whether the Spatial Music Menu could compete with a touch and vision-based music player. During biking the participants had to perform an attention task while exploring and selecting music tracks. Results showed that participants were better at attending the surroundings with the Spatial Music Menu but they used more time navigating to the specific music tracks compaired to the touch and vision-based system. General feedback indicated that participants felt more comfortable using the Spatial Music Menu in terms of perceived workload.

In general the results showed that using systems that are based on alternative modalities like head gestures and audio can decrease the perceptual and cognitive load when interacting while in motion, compaired with systems based on touch and visual modalities. However challenges exists in making such a system as efficient as todays mobile systems. The fact that people have used touch and vision-based interfaces for a while now can have an impact i.e. it will require time for people to learn these "new" interaction modalities and a lot of systems engineering to optimize to the users behaviour.

% other findings
An interesting finding of this thesis is the strength of spatial audio. Users were able to segregate 6-8 simultanous playing sounds in our design process and in the evaluation participants could locate sounds even though they were shuffled. This indicates that the audiospace has a lot of potential when it comes to information presentation and retrieval.


\section{Future Work}
As this thesis evaluation was a controlled lab experiment it could be interesting to see how the Spatial Music Menu would perform in a real world biking scenario. This would however require some engineering as many other factors would suddenly be introduced as described in section \ref{sec:interactioninmotion}.

Although we in this thesis targeted a music player, the auditory menu could be useful in other application types as well. One could imagine music track items being replaced with other non-speech sounds for example a news site with weather news represented by weather sounds, sports news represented by sport sounds, etc. Such kind of application did not have to target interaction on the move - it could for example be suitable for visually impaired people.

Headset controllers are, as described in section \ref{sec:alternativemusicuis}, the preferred choice today when it comes to eyes-free interaction with music players. The evaluation results showed faster navigation when participants used the hand (touch) for navigating, so it could be interesting to see, if a combination of this headset controller and spatial audio output could provide the same kind of auditory menu as the Spatial Music Menu and how it would affect the usability.






% OLD

% Other use than biking scenario

% talk about how more than 3 tracks could be implemented

% solution with a headset controller + spatial audio

%Other scenarios e.g. visual impaired people, car driving...

% Thomas comments
% in general, it should contain:
% 1) reflection over the whole design process (what would you have done different, knowing what you know now (if anything)
% 2) discussion of the main findings of the evaluation (the evaluation chapter has a more detailed discussion). what does it point towards? where is the future of this kind of systems?
% 3) future work (list stuff that was found in eval which could improve future versions of the system, ideas you had yourself but discarded due to lack of resources, etc.) % Conclusion

%% ----------------------------------------------------------------
% Now begin the Appendices, including them as separate files

\addtocontents{toc}{\vspace{2em}} % Add a gap in the Contents, for aesthetics

\appendix % Cue to tell LaTeX that the following 'chapters' are Appendices

\chapter{Evaluation}

\section{Nasa TLX Scales}

\includegraphics[width=\textwidth]{Figures/appendix_nasatlx.png}



	% Appendix Title

%\chapter{iOS Application Screenshots}
\label{sec:appendixviews}

\section{AudioMenu}

\includegraphics[width=0.7\textwidth]{Figures/view_audiomenu.png}


\section{Music}

\includegraphics[width=0.7\textwidth]{Figures/view_music.png}


\section{Gestures}

\includegraphics[width=0.7\textwidth]{Figures/view_gestures.png}



 % Appendix Title

%\chapter{Software}

\section{iOS Application}
\label{sec:appendixios}
The iOS application is on github an can be cloned from: \url{https://github.com/johin/Thesis/tree/master/Application/SpatialMusicMenu}.


\section{Python Biking Simuation System}
\label{sec:appendixpython}
The Python Biking Simulation System (1 script) is on github an can be cloned from: \url{https://github.com/johin/Thesis/tree/master/Evaluation}. % Appendix Title

\addtocontents{toc}{\vspace{2em}}  % Add a gap in the Contents, for aesthetics
\backmatter

%% ----------------------------------------------------------------
\label{Bibliography}
\lhead{\emph{Bibliography}}  % Change the left side page header to "Bibliography"
\bibliographystyle{acm-sigchi}  % Use the "unsrtnat" BibTeX style for formatting the Bibliography
\bibliography{Bibliography}  % The references (bibliography) information are stored in the file named "Bibliography.bib"

\end{document}  % The End
%% ----------------------------------------------------------------
\lhead{\emph{Design}}
\chapter{Design}

\section{Design process}

\subsection{Early prototype testing functionality}

\subsection{Experimental prototype testing menu interaction}


\section{Final prototype}

% prototyping, iterative design process

Plan\\
Iterative design process:\\
- An experimental prototype should be developed and tested. 3 types of menus where a test user should perform head gesture interaction to solve small tasks. User is observed and feedback should be given.\\
- A final prototype should be designed with knowledge from the experimental evaluation. This prototype will go through a "real life" evaluation. It will be evaluated through several days where the user will use the new head gesture based music application and the traditional music application while biking. Again small task could be performed and it could be tested through the users normal use of his/her music application, ending up in a comparison of the traditional vs the new interaction system.

\section{Interaction model}
TODO...\\
Horizontal 180 degrees head movement, nod/shake...

Human head normally can be rotated about 140 degrees for shaking and 100 degrees for nodding \cite{lopresti_neck_2000}.


% Theory keywords: multimodal interaction; mobile hci; 

\section{Sound design}
TODO...\\
Several studies show that circular auditory menus are the way to go because of horizontally positioned sounds , HRTF, 3D audio...

% Theory keywords: audio windows; non-speech sound; audio menus


